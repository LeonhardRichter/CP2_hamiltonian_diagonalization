% Contains an abstract (half a page), a short description of the background of the field and on the simulation method, a detailed description what is planned in the project, a timetable until when the code is expected to be completed and when results are expected

\begin{abstract}
    This report demonstrates the application of the \emph{Lanczos algorithm} for approximating a hermitian matrix by a lower-dimensional tri-diagonal one.
    Given a tri-diagonal approximation on a low-dimensional subspace, standard methods for fully diagonalizing or computing matrix functions such as the exponential become feasible and usually give approximations for the corresponding objects of the original matrix.
    This method is especially efficient when applied to sparse matrices as typically arise in spin Hamiltonians.  
    Here, we consider a system of many non-interacting two-level systems that couple equally to a harmonic oscillator---the Dicke model.
    The Dicke model features a so-called \emph{superradiant phase}, where there are states that spontaneously emit more photons than expected.
    This phenomenon is typically described by a transition amplitude of spontaneous emission scaling up to quadratically with the number of two-level systems.
    Here, we apply Lanczos algorithm in order to simulate dynamical properties of the Dicke model.
    In particular, the expectation value for the number of photons depending on time is compared for different initial states and different coupling strengths.
    Prior to its application to this problem, the algorithm itself is being discussed in detail and is implemented in the programming language \texttt{python}.
\end{abstract}

\section{Introduction}

\begin{itemize}
    \item motivation and background for spin systems
    \begin{itemize}
        \item study of matter and light-matter interaction
        \item look into QFTCM for justification of the Hubbard model
    \end{itemize}
    \item motivation and background for the Lanczos method
    \begin{itemize}
        \item 
    \end{itemize}
\end{itemize}


The time evolution for a duration \(t\in\R\) of a quantum system modeled by a time-independent self-adjoint operator \(H\), the \emph{Hamiltonian}, acting on some Hilbert space \(\HH\) is given by the unitary operator\footnotemark \(U(t) = e^{-\iu t H}\).
\footnotetext{Here, and in the following we work in \carefull{natural units} where \(\hbar = 1\).}




% \section{Introduction}

% \section{Background and methods}

% \subsection{The Dicke-Ising model for light matter-interaction}

% \subsection{The Lanczos method}

% \section{Results}
% \subsection{Time-evolution of the photon number}
% \subsection{}



