% Contains an abstract (half a page), a short description of the background of the field and on the simulation method, a detailed description what is planned in the project, a timetable until when the code is expected to be completed and when results are expected

\begin{abstract}
    This report demonstrates the application of an efficient method for approximately diagonalizing sparse hermitian matrices---the \emph{Lanczos method}.
    An espacially suitable usecase for this method is the application to spin Hamiltonians with only local interactions. 
    Here, an \carefull{Ising model coupled to a monochromatic cavity field--- the Dicke-Ising model--- is being considered and the time evolution of the expected number of photons in the cavity} gets approximated using the Lanczos method.
    This method itself is being discussed in detail and is implemented in the programming language \carefull{\texttt{julia}}.
\end{abstract}

\section{Introduction}

\begin{itemize}
    \item motivation and background for spin systems
    \begin{itemize}
        \item study of matter and light-matter interaction
        \item look into QFTCM for justification of the Hubbard model
    \end{itemize}
    \item motivation and background for the Lanczos method
    \begin{itemize}
        \item 
    \end{itemize}
\end{itemize}


The time evolution for a duration \(t\in\R\) of a quantum system modeled by a time-independent self-adjoint operator \(H\), the \emph{Hamiltonian}, acting on some Hilbert space \(\HH\) is given by the unitary operator\footnotemark \(U(t) = e^{-\iu t H}\).
\footnotetext{Here, and in the following we work in \carefull{natural units} where \(\hbar = 1\).}




% \section{Introduction}

% \section{Background and methods}

% \subsection{The Dicke-Ising model for light matter-interaction}

% \subsection{The Lanczos method}

% \section{Results}
% \subsection{Time-evolution of the photon number}
% \subsection{}



