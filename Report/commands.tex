
% get access to document class options
\makeatletter \usepackage[\@classoptionslist]{optional}

%define a notfinal toggle
\newtoggle{notfinal}	% create a new toggle value from etoolbox
\toggletrue{notfinal}	% default setting is true
\opt{final}{\togglefalse{notfinal}} % if classmode is final, set to false

%define a command for highlighting tentative content. 
%-- Only displayed in draft or normal mode. 
%--Hidden in final mode
\newcommand\texttodo[1]{%
   \iftoggle{notfinal}{%
      {\color{red}#1}%
   }{%
      #1%
   }%
}
\newcommand\carefull\texttodo

% ---------- DICKE ---------

\newcommand\creation{a^\dagger}
\newcommand\ad\creation
\newcommand\adag\creation
\newcommand\annihilation{a}
\newcommand\an\annihilation
\newcommand\photonnumber{\hat{n}}
\newcommand\nn\photonnumber

\newcommand\spin[1]{S^{#1}}
\newcommand\s{\spin}

\newcommand\Splus{\spin+}
\newcommand\Sminus{\spin-}
\newcommand\Sp\Splus
\newcommand\Sm\Sminus
\newcommand\SX{S_x}
\newcommand\Sx{\SX}
\newcommand\sx\SX
\newcommand\SY{S_y}
\newcommand\Sy{\SY}
\newcommand\sy\SY
\newcommand\SZ{S_z}
\newcommand\Sz{\SZ}
\newcommand\sz\SZ

\newcommand\pauli[1]{\sigma_{#1}}
\newcommand\pauliI[2]{\pauli{#1}^{(#2)}}
\newcommand\paulii\pauliI


% ---------- MATH ----------

\newcommand{\iu}{\mathrm{i}\mkern1mu}
\renewcommand{\Re}{\mathfrak{Re}}
\renewcommand{\Im}{\mathfrak{Im}}

\newcommand{\id}{\mathrm{id}}
\newcommand{\primed}{^\prime}
\newcommand{\primedd}{^{\prime\prime}}
\newcommand{\dual}{^\prime}

\DeclareMathOperator{\one}{\mathds{1}}
\newcommand{\adj}{^\ast}
\newcommand{\inv}{^{-1}}
\newcommand\close\overline

\newcommand\Ltwo[1]{L^2(#1)}


\newcommand{\HH}{\mathcal{H}}
\newcommand{\N}{\mathds{N}} %natural numbers
\newcommand{\Z}{\mathds{Z}} %integers
\newcommand{\Q}{\mathds{Q}} %rationals
\newcommand{\R}{\mathds{R}} %reals
\newcommand{\C}{\mathds{C}} %complex numbers


\DeclareMathOperator{\vecspan}{span}

\renewcommand{\Re}{\mathfrak{Re}}
\renewcommand{\Im}{\mathfrak{Im}}

\newcommand{\dd}{\odif} % needs package 'derivative' 

% ---------- Bracktes ----------


% Klammern - mit Variablen Größen
% #1 gives the variable or a dot if the arguement is left empty. Requeire etoolbox
\DeclarePairedDelimiter{\of}{(}{)}
\DeclarePairedDelimiter{\cof}{\{}{\}}
\DeclarePairedDelimiter{\eof}{[}{]}
\DeclarePairedDelimiterX\comm[2]{[}{]}{#1, #2}
\newcommand\commutator\comm
\DeclarePairedDelimiterX\innerp[2]{\langle}{\rangle}{#1,\; #2}
\DeclarePairedDelimiterX\braket[2]{\langle}{\rangle}{#1\,\delimsize\vert\, #2}
\DeclarePairedDelimiterX\ket[1]{\lvert}{\rangle}{#1}

%\usepackage{etoolbox}
\newcommand{\emptyplaceholder}{\phantom{A}} % alternatively \;\cdot\;
\DeclarePairedDelimiterX\norm[1]\lVert\rVert{
    \ifblank{#1}{\emptyplaceholder}{#1}
}

\DeclarePairedDelimiterXPP\pnorm[1]{}\lVert\rVert{_{p}}{
    \ifblank{#1}{\emptyplaceholder}{#1}
}

\DeclarePairedDelimiterXPP\twonorm[1]{}\lVert\rVert{_{2}}{
    \ifblank{#1}{\emptyplaceholder}{#1}
}

\DeclarePairedDelimiterXPP\opnorm[1]{}\lVert\rVert{_{\mathrm{op}}}{
    \ifblank{#1}{\emptyplaceholder}{#1}
}


\DeclarePairedDelimiterX\abs[1]\lvert\rvert{#1}

\providecommand\given{}
\DeclarePairedDelimiterXPP\Prob[1]{\mathcal{P}}(){}{
   \renewcommand\given{\nonscript\:\delimsize\vert\nonscript\:\mathopen{}}
   #1}

% just to make sure it exists
\providecommand\given{}
% can be useful to refer to this outside \Set
\newcommand\SetSymbol[1][]{%
   \nonscript\:#1\vert
   \allowbreak
   \nonscript\:
   \mathopen{}}
\DeclarePairedDelimiterX\Set[1]\{\}{%
\renewcommand\given{\SetSymbol[\delimsize]}
#1
}

\newtheorem{lemma}{Lemma}